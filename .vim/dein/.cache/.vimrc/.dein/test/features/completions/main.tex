\documentclass{article}
\usepackage{amsmath}
\usepackage{amsthm}
\usepackage{cleveref}
\usepackage{graphicx}
\usepackage{pdfpages}
\usepackage{standalone}

\usepackage[backend=biber,style=authoryear]{biblatex}
\addbibresource{biblatex-examples.bib}
\addbibresource{local1.bib}
\addbibresource{~/.vim/bundle/vimtex/test/features/completions/local2.bib}

\usepackage[nonumberlist,acronymlists={gloss,symbolslist}]{glossaries}
\loadglsentries{glossary}

\newtheorem{theorem}{Theorem}
\newtheorem{proposition}{Proposition}

\includeonly{sub1,sub2 with spaces}

\begin{document}

To test label completion you need to compile the document first!

\begin{equation}
  f(x) = 42
  \label{eq:main-is-working}
\end{equation}

\begin{theorem}
  \label{th:first}
  test
\end{theorem}

\begin{theorem}
  \label{th:second}
  test
\end{theorem}

\begin{proposition}
  \label{prop:first}
  test
\end{proposition}

\begin{proposition}
  \label{prop:second}
  test
\end{proposition}

\section{sub1}\label{sec:sub1}

\begin{equation}
  f(x) = 42
  \label{eq:sub-is-working}
\end{equation}


\include{"sub2\space with\space spaces"}

\input{"sub3\space with\space spaces"}

\begin{equation}
  \label{eq:test1}
  f(x) = 1
\end{equation}

\begin{equation}
  \label{eq:test2}
  f(x) = 2
\end{equation}

\begin{equation}
  \label{eq:test3}
  f(x) = 3
\end{equation}

\begin{equation}
  \label{eq:test4}
  f(x) = 4
\end{equation}

\begin{equation}
  \label{eq:test5}
  f(x) = 5
\end{equation}

\begin{equation}
  \label{eq:test6}
  f(x) = 6
\end{equation}

\begin{equation}
  \label{eq:test7}
  f(x) = 7
\end{equation}

\begin{equation}
  \label{eq:test8}
  f(x) = 8
\end{equation}

\begin{equation}
  \label{eq:test9}
  f(x) = 9
\end{equation}

\begin{equation}
  \label{eq:test10}
  f(x) = 10
\end{equation}

\begin{subequations}
  \begin{align}
    \label{eq:test11a}
    f(x) &= 11a \\
    \label{eq:test11b}
    f(x) &= 11b
  \end{align}
\end{subequations}


% Examples of Biblatex citation commands
\cite{aristotle:physics} \\
\cite[5--10]{aristotle:physics} \\
\cite[see][5--10]{aristotle:physics} \\
\parencite{aristotle:physics} \\
\parencite[5--10]{aristotle:physics} \\
\parencite[see][5--10]{aristotle:physics} \\
\textcite{aristotle:physics} \\
\textcite[5--10]{aristotle:physics} \\
\textcite[see][5--10]{aristotle:physics} \\

equation (\ref{eq:main-is-working}) \\
equation \eqref{eq:sub-is-working} \\
\cref{eq:test1,eq:test2} \\
\crefrange{eq:test1}{eq:test10} \\
\cref{eq:inputted sub with spaces is working} on \cpageref{sec:sub3 with spaces} \\
\cref{sec:sub1,sec:sub3 with spaces} \\
\crefrange{sec:sub1}{sec:sub3 with spaces} \\

\newpage
\glsaddall
\printglossary[
  type=gloss,
  style=long,
  title={Glossary},
  toctitle={Glossary}
]
\printglossary[
  type=symbols,
  style=long,
  title={List of Symbols},
  toctitle={List of Symbols}
]

\end{document}
